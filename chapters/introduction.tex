\section{研究背景及意义}

随着信息化时代的加速发展,图像作为一种高效直观的信息载体,在人类生活的多个方面展示了其独特的重要性。数字图像是图像的信息化形式, 由于其数字化的特性和可被编程的能力而被广泛应用。 然而数字图像在采集、传输、存储、处理以及显示的过程中, 可能由多种因素导致数字图像中出现丢失的、错误的数字信号,与图像想表达的真实信息不符,这种干扰被称之为图像噪声。

图像去噪是任何图像处理流程中的重要组成部分,在许多重要领域都有应用,包括生物医学成像\cite{ctdenoise, eformer}、遥感成像 \cite{hybridhsidenoise, cnnhsidenoise}、暗光摄影成像\cite{eld}、农业图像\cite{鸡儿denoise, 杂交denoise}等。在进行各种类型的更复杂的任务和算法(例如人脸识别\cite{qualnet}、图像分类\cite{ddp, sfdunet}等)之前,去噪通常是预处理数据集的重要步骤。一个好的去噪模型的要求是在去除图像噪点的同时保留边缘、纹理区域和其他图像细节,在消除噪点和保留精细图像细节之间存在着复杂的平衡,这使消除图像中的噪点成为了一个具有挑战性的问题。此外,噪声源的多变性和不可预测性,图像不断增长的分辨率,也都是开发有效降噪算法所面临的挑战。

在进入深度学习的时代之前,许多去噪算法依赖于贝叶斯公式,即对图像噪声先验的依赖。图像噪声先验构成了图像处理整体进展的支柱,这条路径一般称之为传统图像去噪算法。广泛的工作引入了许多性能良好的去噪算法,以及一个计算机视觉中引人注目的的科学领域。但事实上,这些光辉的成就仍使一些研究人员开始考虑“去噪已死”的可能性\cite{去噪已死},似乎现有的解决方案已经触及了可实现的性能上限。然而近年来,深度学习技术的快速发展为图像去噪提供了新的解决方案,其中也包括图像噪声先验的新发展。研究者们通过深度神经网络强大的特征提取和模式识别能力,从大量的图像数据中学习到噪声的分布和特性,从而实现更为准确和自适应的噪声参数估计,进而进一步提升去噪算法的性能。

%从早期的基于$L_2$的正则化开始,到小波域的引入,再到对偏微分方程的部署,这其中一直延续了稀疏编码、基于补丁的方法和低秩结构假设的方法。

随着社会逐渐进入智能化时代,图像作为机器认识世界的眼睛,使得社会对图像质量的要求越来越高,图像去噪技术也被越来越多的应用。因此,本文对基于深度学习的图像噪声先验学习与去噪方法展开更深入的探究,提出去噪效果更优秀的去噪算法具有重要的实际意义与使用价值。

\section{研究现状}

\subsection{噪声先验}

噪声先验学习,即对真实的噪声进行尽可能准确的建模,从而获得噪声的先验知识。现有的噪声建模方法主要分为两类:一种是基于物理的噪声模型,另一种是基于深度卷积神经网络的噪声模型。

基于物理的噪声模型通过分析相机成像管道的物理过程,获得各种噪声源的统计分布。最广泛采用的噪声模型是与信号无关的加性高斯白噪声(AWGN),即$n_i \sim \mathcal{N}\left(0,\sigma^2\right)$,其中$n_i$代表了噪声在位置$i$处的强度值,$\sigma$代表了这个假设下噪声的标准差,但实际上高斯白噪声与现实世界中的复杂噪声相去甚远 \cite{nlm}。为了解决噪声与图像信号无关的问题,研究者们提出了泊松高斯噪声模型(Poisson-Gaussian,P-G)\cite{foi2008noise, foi2009noise},这一模型解释了噪声与图像信号的相关性。为了让噪声模型更加通用和灵活,高斯混合模型\cite{noisemodeling2denoising}被提出,它可以逼近任意连续分布。近年来,研究人员也已经开发了更复杂的噪声模型 \cite{eld, rethinking, awgn},希望通过校准传感器特定参数来近似真实的噪声分布。然而,数据校准过程是耗时费力的,这阻碍了基于物理的噪声建模方法的发展。以及传感器参数分布间往往存在着显著差异,这进一步挑战了结果的可靠性。

基于深度卷积神经网络的噪声建模方法学习以数据驱动的方式,借助深度生成网络来合成噪声\cite{noiseflow, learningcamera, c2n, gcbd, dan}。基于生成对抗网络的的盲去噪器(GAN-CNN based Blind Denoiser,GCBD)提出了一种噪声建模框架,首次尝试通过利用生成对抗网络来合成真实的噪声图像。分组残差稠密网络(Grouped residual dense network ,GRDN)\cite{grdn},提出利用条件信息(如干净图像、ISO、快门速度和相机传感器)来合成真实噪声。噪声流(Noise Flow)\cite{noiseflow}方法使用标准化流来显式地生成原始噪声。尽管这些方法在表征复杂的真实世界噪声方面已经取得了显著的效果,但它们仍然存在一些缺点,包括对成对噪声干净数据集的依赖性,以及训练的不稳定性\cite{srgbflow}。此外,这些方法往往会过度简化相机成像管道\cite{co}。

作为一种数学框架上的概率模型,统计噪声模型往往对噪声源有着更加深入理解,往往有助于更好地处理和校正图像中的噪声。因此,在实际应用中有必要将物理先验纳入数据驱动方法,以平衡模型的复杂性和适用性。

\subsection{图像去噪}

图像去噪算法自上世纪六十年代发展至今, 基本可分为两大类:传统的图像去噪和基于深度学习的图像去噪。

传统的图像去噪方法则又可以分为两类,包括基于滤波的方法和基于模型的方法。基于滤波的方法大致可分为空间域去噪方法\cite{nlm}和变换域去噪方法\cite{waveletdenoising}。而结合空间域与变换域的思想诞生了最经典的基于滤波的方法, Dabov等人在2007年提出了BM3D 算法(Block-Matching 3D algorithm)\cite{bm3d},算法的核心思想是将图像分成许多块,然后进行块匹配,将相似的二维块聚合成三维块,通过对这些三维块进行变换和滤波去除噪声。基于模型的去噪算法通常会将噪声视为图像中的另一种数据, 通常基于一些假设利用数学模型对图像进行去噪处理,包括变分模型\cite{totalvariation}、 稀疏表示\cite{ksvd}和图像低秩性\cite{wnnm}等。

%空间域去噪算法通过使用可以在像素间移动的滤波器模板,对图像像素值直接进行加权平均、取中值、滤波等操作,来实现图像去噪。1998年,Tomasi等人提出了双边滤波\cite{bilateral},这是一种结合图像的空间邻近度与像素值相似度的处理办法,在滤波时,该滤波方法同时考虑空间临近信息与颜色相似信息,在滤除噪声、平滑图像的同时,又做到边缘保存。2005年Buades等人提出了非局部均值滤波算法(Non-local Means,NLM)\cite{nlm},与基于像素点周围的局部邻域滤波算法不同,NLM 算法在整张图中搜索与像素点最相似的像素块, 并对这些像素块进行加权平均得到该像素点的像素值。变换域去噪算法的核心思想是通过将图像转换为频域,在频域中去除图像中的噪声信号,然后将处理后的频域信号转换回空间域,得到去噪后的图像。其中代表算法包括傅里叶变换去噪算法, 小波变换去噪算法等。 傅里叶变换将信号分解为一系列正弦函数,因此能够非常有效地去除周期性的噪声;小波变换将信号分解为一系列小波,每个小波具有不同的频率和时间间隔,因此能够有效地去除不同频率的噪声。

 %1992年,Rudin等人提出了经典的全变分去噪算法\cite{totalvariation},该算法假设图像平滑, 最小化图像的全变分来去除图像噪声。2006年,Elad等人年提出了K-SVD\cite{ksvd}去噪算法,该算法是经典的基于稀疏表示的图像去噪算法, 该算法通过迭代自适应学习字典, 更好的适应不同图像以及各种噪声强度。2014年,Gu等人提出的加权核范数最小化去噪方法(Weighted nuclear norm minimization,WNNM)\cite{wnnm},利用图像非局部自相似性解决去噪问题。

自从DnCNN\cite{dncnn}首次将卷积神经网络应用于图像去噪以来,大量基于深度学习的去噪方法\cite{ffdnet, focnet, restormer, swinir}被开发出来,并取得了令人印象深刻的性能。然而,多数现有的深度学习方法基于这样一个假设,即图像所含噪声是可以被预定义的,例如加性高斯白噪声(AWGN)。但在实际场景中捕获的噪声远比高斯噪声复杂,这导致这些方法在实际应用中甚至不如传统方法。因此,在未给出准确的噪声分布和参考信息的情况下进行去噪,或者称为盲去噪,已成为一个重要且具有挑战性的课题。

为了解决这一问题,很多研究者已经进行了深入的研究。CBDNet\cite{cbdnet}提出了一个基于异方差高斯噪声模型的噪声水平估计模块。Path-Restore方法\cite{pathrestore}采用动态策略,为每个噪声区域选择合适的路径。近年来,自监督去噪方法受到了更多的关注,Noise2Noise\cite{noise2noise}和Noise2Void\cite{noise2void}等方法表明,仅通过在噪声图像上进行训练就可以实现去噪,这在实际应用中具有重大意义。然而,这些方法也存在局限性,例如对特定类型噪声的敏感性,因此,开发一种鲁棒且灵活的盲去噪方法是具有重要价值的。

\section{论文的研究内容及章节安排}

\subsection{研究内容}

基于深度学习的去噪方法在消费级相机或移动设备拍摄的真实照片上会出现急剧的性能下降。这种现象主要是由于噪声假设与现实噪声分布之间的较大偏差,导致训练数据与测试数据之间存在域上的差距。因此,在现实图像去噪的领域中,噪声先验学习对于实现高质量的去噪效果至关重要。为了解决上述问题,本文将基于物理的噪声模型与深度生成网络相结合,不直接合成噪声图像,而是利用生成网络来生成基于物理的噪声模型的参数。具体来说,首先利用传感器校准数据来预训练标准化流模型,将该模型作为生成先验,从简单的高斯分布中合成噪声参数。为了使生成器能够更准确地模拟真实世界的噪声图像,本文利用对抗学习来区分噪声的特征。

本文的工作驱动主要有三个方面。一方面,现有的生成式模型忽略了物理噪声模型的先验信息,忽略了传感器成像过程中噪声的形成,这限制了模型无法去生成更真实的噪声。另一方面,基于深度神经网络的噪声建模方法需要成对的噪声干净图像数据集,这在实际应用中很难获得。此外,不同的传感器需要特定的参数校准,导致训练的网络泛化能力较差。本文的研究工作总结如下:

(1)本文提出了一种新颖的生成式噪声建模方法,利用预训练的基于流模型的参数先验模型来生成噪声。该方法通过使用潜在空间生成噪声参数,而不是直接合成复杂的噪声,提高了噪声表征的准确性,使得本文的方法在各种应用中的噪声模拟更为稳健和高效。

(2)本文的方法通过标准化流将复杂的噪声参数场表示为简单的高斯分布,称之为基于流模型的噪声参数先验。这种方法显著压缩了物理噪声参数所需的采样空间,提高了建模的计算效率和精度。

(3)本文提出的灵活噪声建模框架可以根据不同的物理参数噪声模型进行动态配置。通过采用对抗训练技术,该框架无需配对的噪声-感觉图像训练数据,从而简化了训练过程并提高了效率。



\subsection{章节安排}

第一章阐述了噪声先验、图像去噪等的研究背景和基本特点,介绍了将噪声先验信息应用到图像恢复领域的研究意义。之后对噪声先验以及图像去噪的国内外研究现状做了系统介绍。

第二章阐述了本文的相关工作,包括相关的图像去噪方法、标准化流模型、流模型在底层视觉领域的相关应用以及相关的噪声模型。

第三章系统介绍了本文所提出的基于标准化流的噪声建模,首先是噪声在物理上的形成模型,之后是基于流模型的噪声参数先验,最后是基于参数先验的噪声建模。

第四章将本文提出的图像噪声先验学习方法应用到去噪方法当中,与别的噪声模型相对比,与别的去噪方法相对比。

第五章对本课题做了总结分析,对未来可能的研究发展做出展望。

