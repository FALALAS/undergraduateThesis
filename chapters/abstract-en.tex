Image denoising is a classic low-level vision problem aimed at recovering unknown clean images from noisy observations. In research this problem is usually categorized into two classes, mainly distinguished in the presence or absence of a noise parameter prior; image denoising tasks without a noise parameter prior are generally referred to as blind denoising tasks. Under the assumption of a known noise prior, both traditional and deep learning methods have achieved impressive performance. However, since the training dataset and the real image data are often far from each other, these methods suffer a dramatic performance drop on blind denoising tasks, which highlights the importance of noise prior learning.

Some existing methods attempt to simulate noisy images by analyzing possible degradations in the signal processing pipeline of a particular camera image to obtain a physical parametric noise model, but they often struggle to fully capture complex noise distributions and lack generalizability. Deep learning-based paradigms commonly utilize deep generative networks to synthesize noise, but these data-driven approaches often require paired noise-clean image datasets. Researchers' experience has demonstrated that it is extremely difficult to produce such datasets from real images, which poses a challenge to the practical application of deep learning-based methods in real-world scenarios.

To address these issues, this thesis proposes a new approach that utilizes a deep neural network to infer the parameters of a physically-based noise model, rather than directly synthesizing the noise itself. Specifically, in order to improve the accuracy of parameter estimation, this thesis represents the noise parameter field in the latent space via normalizing flows, called the flow-based noise parameter prior. Using the pre-trained noise prior, the noise parameters can be generated from the underlying Gaussian distribution. After generating the noise parameters, this thesis introduces a physics-based noise parameter model to synthesize the noise from the generated noise parameters. To further improve the realism of the synthetic noise images, the method in this thesis is optimized using adversarial learning techniques to ensure that these images are more similar to real noise images encountered in practical settings. The method in this thesis is then used as a noise pattern prior to generate paired noise-clean datasets, which are used as a basis for training denoising neural networks.

Extensive experimental results demonstrate that the noise synthesized by this thesis' noise prior learning method reaches the state-of-the-art in terms of both subjective and objective quality, and demonstrates significant advantages in blind denoising applications, which not only significantly improves the denoising effect, but also reveals the practicability and versatility of the method in practical applications.