图像去噪是一个经典的底层视觉问题,旨在从噪声观测中恢复未知的干净图像。研究中这个问题通常被分类两类,主要区分在有无噪声参数先验,没有噪声参数先验的图像去噪任务一般被称为盲去噪任务。在已知噪声先验的假设下,无论是传统方法还是深度学习方法都已经取得了令人印象深刻的性能。然而,由于训练数据集和真实图像数据往往相去甚远,这些方法在盲去噪任务上会出现显著的性能下降,这就凸显了噪声先验学习的重要性。

现有的一些方法试图通过分析特定相机图像信号处理流程中可能出现的退化来得到物理上的的参数噪声模型,进而模拟噪声图像,但它们往往难以完全捕捉复杂的噪声分布且缺乏泛用性。基于深度学习的范式普遍利用深度生成网络来合成噪声,但这些数据驱动的方法往往需要配对的噪声-干净图像数据集。研究者们的经验已经证明,从真实图像中制作这种数据集是极为困难的,这对基于深度学习的方法在现实场景中的实际应用提出了挑战。

为了解决这些问题,本文提出了一种新的方法,该方法利用深度神经网络来推断基于物理的噪声模型的参数,而不是直接合成噪声本身。具体来说,为了提高参数估计的准确性,本文通过标准化流模型来表示潜在空间中的噪声参数场,称为基于标准化流模型的噪声参数先验。利用预训练的噪声先验,可以从基本高斯分布生成噪声参数。在生成噪声参数之后,本文引入了一个基于物理的噪声参数模型,通过生成的噪声参数合成噪声。为了进一步提高合成噪声图像的真实性,本文的方法采用了对抗学习技术进行优化,确保这些图像与实际设置中遇到的真实噪声图像更加相似。之后本文的方法将作为噪声模式先验来生成成对的噪声-干净数据集,以此为基础训练去噪神经网络。

广泛的实验结果表明,本文的噪声先验学习方法合成的噪声在主观和客观质量方面都达到了最先进的水平,在真实图像去噪应用中展现了显著的优势,不仅显著提高了去噪效果,还显现出该方法在实际应用中的实用性和通用性。
